%! Author = soheili
%! Date = 9/5/2022

% Preamble
\documentclass[11pt]{article}

% Packages
\linespread{1.5} % normal line space
\usepackage{amsmath}
\usepackage{setspace}
\usepackage{hyperref}
\usepackage[hscale=0.7,vscale=0.8]{geometry}
\usepackage{graphicx}
\usepackage{amssymb}
\usepackage{cite}
\usepackage[ruled,lined,linesnumbered]{algorithm2e}
\usepackage[inline]{enumitem}
\usepackage{subcaption}
\usepackage{algpseudocode}
\usepackage{multirow}
\usepackage{microtype}
\usepackage{amsfonts}
\usepackage{color}
\DisableLigatures{encoding = *, family = *}

% Document
\begin{document}
    \title{BiAS: Bioinformatic Attributes Standardization - Documentation}
    \author{Majid Soheili}
    \maketitle

    \section{Introduction}\label{sec:intro}
    In the Bioinformatics world, hundreds of studies and research are performed daily.
    Most of them prepare their data to public in public repository like NCBI.\@
    Although there are some minimum standard for publishing datasets, but the authors can define some custom attributes by themselves.
    These attributes are saved in a same format {name:value}.
    Sometimes different authors use various name for a single feature.
    For example, for describing of the gender of the host, various names such as gender and sex can be used by authors.
    In this project, aggregating the same attributes that probably can have different names is a major aim.
    However, ahead problems are more complicated than having different names for an individual attribute.

\end{document}