%! Author = soheili
%! Date = 9/5/2022

% Preamble
\documentclass[11pt]{article}

% Packages
\linespread{1.5} % normal line space
\usepackage{amsmath}
\usepackage{setspace}
\usepackage{hyperref}
\usepackage[hscale=0.7,vscale=0.8]{geometry}
\usepackage{graphicx}
\usepackage{amssymb}
\usepackage{cite}
\usepackage[ruled,lined,linesnumbered]{algorithm2e}
\usepackage[inline]{enumitem}
\usepackage{subcaption}
\usepackage{algpseudocode}
\usepackage{multirow}
\usepackage{microtype}
\usepackage{amsfonts}
\usepackage{color}
\DisableLigatures{encoding = *, family = *}

% Document
\begin{document}
    \title{BiAS: Bioinformatic Attributes Standardization - Documentation}
    \author{Majid Soheili}
    \maketitle

    \section{Introduction}\label{sec:intro}
    In the Bioinformatics world, hundreds of studies and research are performed daily.
    Most of them prepare their data to public in public repository like NCBI.\@
    Although there are some minimum standard for publishing datasets, but the authors can define some custom attributes by themselves.
    These attributes are saved in a same format {name:value}.
    Sometimes different authors use various name for a single feature.
    For example, for describing of the gender of the host, various names such as gender and sex can be used by authors.
    In this project, aggregating the same attributes that probably can have different names is a major aim.
    However, ahead problems are more complicated than having different names for an individual attribute.
    
    \section{Previous Works}\label{sec:previouswork}
    Kasmanas et al.\@ published a paper in about the standardization metadata for human meta-genomes datasets.

    \section{NCBI Dataset Structures}\label{sec:ncbi}
    National Center for Biotechnology Information (hereafter NCBI) is part of the United States National Library of Medicine.
    It is approved and funded by the government of the United States.\@
    The NCBI houses a series of databases relevant to biotechnology and biomedicine and is an important resource for bioinformatics tools and services.
    Sequence Read Archive (SRA) data, available through multiple cloud providers and NCBI servers, is an archive for high throughput sequencing data, publicly accessible to enhance reproducibility in the scientific community.
    There are four hierarchical levels of SRA entities and their accessions:
    \begin{enumerate}
        \item Study: with accessions in the form of SRP, ERP, or DRP.\@
        \item Sample: with accessions in the form of SRS, ERS, or DRS.\@
        \item Experiment: with accessions in the form of SRX, ERX, or DRX.\@
        \item Run: with accessions in the form of SRR, ERR, or DRR.\@
    \end{enumerate}
    The noticeable point is that the minimum publishable unit in the SRA, is an Experiment.

\end{document}